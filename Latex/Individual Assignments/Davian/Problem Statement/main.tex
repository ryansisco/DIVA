\documentclass[letter, 10pt, titlepage, onecolumn, draftclsnofoot]{IEEEtran}

\usepackage{color}
\usepackage{geometry}
\usepackage{graphicx}
\usepackage{amsthm}
\usepackage{amssymb}
\usepackage{amsmath}
\usepackage{float}
\usepackage{balance}
\usepackage{enumitem}
\usepackage{pstricks, pst-node}
\usepackage{hyperref}
\usepackage{url}

\geometry{textheight=8.5in, textwidth=6in}

\def\name{Davian Lukman}
\author{\name}
\title{CS461 Senior Capstone Fall 2018\newline 3D Visualization of Data}

\hypersetup{
  colorlinks = true,
  urlcolor = black,
  pdfauthor = {\name},
  pdfkeywords = {cs461''senior capstone''problem statement''draft},
  pdfpagemode = UseNone
}

\begin{document}
\maketitle
\hrulefill

\begin{abstract}
\noindent Scientists are often getting problems to leave a mark for their audiences in presenting their data. The purpose of this project is to display an interesting method of presenting data that could be used to satisfy the audiences and also helping scientist to find new possibilities. The solution of this problem is to make an application for the researchers to input their data easily and visualize them in 3D. The first test method that needs to be described is writing a problem statement to fulfill the requirement of the applications. The second step is to design a user interface to let the researcher easily input their data and timeline. The third step is to implement the program based on usability problem and requirement. The results will be delivered to the clients through presentation and implementation. The program will likely to have bugs and it is recommended to be re-examined to fully polish the program.
\end{abstract}

\emph{PROBLEM STATEMENT}
\newline 
The problem of this project is how to display data for in interesting and challenging way to client focused media, such as PowerPoint, Blogs, or Video. Before going on further, the definition of data is needed, which includes IP-Address, hash of malware, name of malware, Geo-Location, tag, time, etc. combined in a file called CSV. Displaying data to clients can somewhat be difficult if visualize incorrectly or without proper descriptions. It becomes harder when the clients are not available in person as presentation can only occur using media. Since the clients have to know what they are reading without being guided, finding a way to present a data in a new way could be challenging. 
\newline 
On the other hand, there are actually many applications to display data, but not all of them actually uses an interesting and new way to display data or perhaps not enough information provided. The usual data presentation are shown using charts or tables. This is a usual way to present a data due to its simplicity; however, it proves to be hard to read when the data is big enough. For example, using tables with too many columns can sometimes confuses the reader to misread one or two rows above or below. Since timeline and Geo-location input are two of the requirement of this project, using time line chart is not a good way to demonstrate data presentation. 

\emph{SOLUTION}
\newline
The solution that can be offered right now is designing a 3D visualization of data presentation. 3D visualization is important because the data includes location, which will be accurately presented on a globe. Since this is related to huge amount of data, a database is required to sort all data. Database can be used to relate other data in the repository. For example, key data would include IP-Address, hash of malware. IP-Adress will be linked to user's name and location of the user, while hash of malware will be linked to name of malware. Another solution can also be achieved with just reading CSV file with buffer and read it and store it to the memory. For the timeline solution, usage of side scroll-bar can be used to properly displaying a certain time. Perhaps, a feature to compare two or different timelines is also necessary. A line will be displayed around the globe to show connections the spreading of the malware.
\newline
Another solution that can be offered now is a use of 3D line graph inspired by human's neuron cell. This solution removes the Globe User Interface features to determine the location of the IP-Address; however, this line graph will show nodes that will tell the key data and can be clicked to open another window of the complete data. The use of this graph excels in its time feature as the users can zoom in and zoom out to see the connection between specific timelines. The lines should also define the spreading of the malware. Same with the first solution, this solution requires the use of database or read buffer of CSV file.

\emph{PERFORMANCE METRICS}
\newline
The project is completed when an interactive 3D visualization of data can read CSV file and stored in the database or variables. Interactive 3D visualization will includes timeline and geo-location to improve the interest of the audiences. A good design of the User Interface will be the secondary objective of this project to further enhance presentation experience for the users.


%Reference
\nocite{*}
\bibliographystyle{IEEEtran}
\end{document}