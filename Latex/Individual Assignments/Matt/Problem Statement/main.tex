\documentclass[letterpaper,10pt,titlepage, onecolumn]{IEEEtran}

\usepackage{graphicx}                                     
\usepackage{amssymb}                                       
\usepackage{amsmath}                                       
\usepackage{amsthm}  

\usepackage{alltt}                                         
\usepackage{float}
\usepackage{color}
\usepackage{url}

\usepackage{balance}
\usepackage[TABBOTCAP, tight]{subfigure}
\usepackage{enumitem}
\usepackage{pstricks, pst-node}

\usepackage{geometry}
\geometry{textheight=8.5in, textwidth=6in}

\newcommand{\cred}[1]{{\color{red}#1}}
\newcommand{\cblue}[1]{{\color{blue}#1}}

\usepackage{hyperref}
\usepackage{geometry}

\def \name{Bhavya Parikh}

\hypersetup{
  colorlinks = true,
  urlcolor = black,
  pdfauthor = {\name},
  pdfkeywords = {CS461 ``Senior Software Engineering Project'' Problem Statement},
  pdftitle = {CS 461 Problem Statement},
  pdfsubject = {CS 461 Problem Statement},
  pdfpagemode = UseNone
}

\begin{document}

\begin{titlepage}
\begin{center}
  
  \textbf{}

  \vspace{6cm}
  \Huge{}
  \textbf{Problem Statement}
  \vspace{3cm}

 
  \LARGE
  CS461 - Senior Software Engineering Project\\
  \vspace{0.25cm}
  Instructor: D. Kevin McGrath \\
  Instructor: Kristen Winters \\
  \vspace{0.25cm}
  Fall 2018 \\
  \vspace{1.5cm}
  
  \textbf{Matthew L. Jansen}
  \date{October 10th, 2018}
  \vfill
  October 10th, 2018\\
  \vspace{1cm}
  \vspace*{\fill}
  \textbf{Abstract}
  \end{center}
This paper will look at the challenge of presenting new data in a meaningful and unique way in the form of an application. We will describe the problem in depth, and the implications that it has in industry and academia. Then, we will propose a detailed solution to the issue using an application to visualize data in a 3D environment, which will map not only the objects within each file, but their relationships as well. Lastly, we will define performance metrics to measure the applicability and accuracy of the application.
  \end{titlepage}
  
\section*{DESCRIPTION OF THE PROBLEM}
One reoccurring challenge in a variety of fields is the ability to present and visualize data in new ways. This is especially the case for large datasets which can be difficult to visualize, let alone find relationships or patterns in. Humans are able to interpret datasets more accurately and completely when presented in the form of a complex visual image, compared to reading and sorting through several log files, attempting to find correlations and draw conclusions by hand. 

Furthermore, given certain datatypes such as IP address and their corresponding geo-locations, names and hashes of malware, and other various tags, it can be helpful to visualize these objects in a 3D environment. Without the ability to view data in this format, it can be difficult to see how objects within a data set interact with each other and define how internetworks of these objects might operate in a given context or environment.

 This can have a negative impact in both industry and academia, as it prevents a wealth of knowledge from being brought forward to their respective customers and researchers. This pitfall can also have negative impacts in a time-constrained environment, such as when conducting network forensics to determine the source of unwanted malware within a network, and identifying it accordingly based off fingerprints. 

\section*{PROPOSED SOLUTION}
Our proposed solution is an application which takes a CSV file as input and is able to produce a 3D environment in conjunction with a relevant timeline. This 3D environment depicts each object presented in the file, as well as its relationship to other objects within the environment. 

This simplistic approach of analyzing and graphing CSV files will make it easier for its users to utilize and understand the application, while removing the need to require and process complex data sets as input. Additionally, the integration of a timeline will provide the user not only with a snapshot of their data, but a story of how the data evolved over time.

\section*{PERFORMANCE METRICS}
In order to define how well our application performs and decide completion of this project, performance metrics need to be created, measured, and fulfilled. 

The first metric we will apply is the applications ability to correctly label and parse the input from the CSV file. Specifically, this measures how the application will split the CSV file into individual objects and identify the datatype that is being read. This will be measured by testing the application with several different CSV files, containing IP addresses, geo-locations, malware hashes, and names. The output of each test will determine if the application was able to parse and label each object in the CSV file correctly and completely.

The second metric we will test is how well the application is able to correlate an object’s existence to another object, or to find a specific relationship between two unique objects. Specifically, this will test how well the application can find relationships in a variety of datatypes or data sets and determine what kind of relationship it might be (existence based, communication, geo-location proximity, etc.).

The third metric we will test is the applications ability to form the parsed objects and correlations into a timeline. Specifically, this entails mapping the development of the data set into a set of snapshots, which then are compiled into a story of the evolution of data. Furthermore, this will test the applications ability to determine how many objects to include in a specified snapshot of the story, by analyzing the time intervals between each object’s appearance and location within the given input. This metric will be tested by providing multiple CSV files with similar object properties, but differing time intervals and appearances within the file. The output of each test should correspond with the differences in time intervals within the CSV files. 

The fourth metric will test the applications ability to develop a context or environment that is relevant to the parsed objects. Specifically, this entails including properties into the output such as output scope and geo-locations correctly in the form of a globe. Doing so has the benefit of identifying outliers and creating a better understanding for the user. Additionally, this includes correctly implementing a 3D view of the data. 

\end{document}