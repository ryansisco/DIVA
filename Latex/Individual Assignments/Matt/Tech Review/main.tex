\documentclass[letterpaper,10pt,titlepage, onecolumn, draftclsnofoot]{IEEEtran}

\usepackage{color}
\usepackage{geometry}
\usepackage{graphicx}
\usepackage{amsthm}
\usepackage{amssymb}
\usepackage{amsmath}
\usepackage{float}
\usepackage{balance}
\usepackage{enumitem}
\usepackage{pstricks, pst-node}
\usepackage{hyperref}
\usepackage{url}

\hypersetup{
  colorlinks = true,
  urlcolor = black,
  pdfkeywords = {CS461``Senior Software Engr Project''Tech Review},
  pdftitle = {CS 461 Tech Review},
  pdfsubject = {CS 461 Tech Review},
  pdfpagemode = UseNone
}

\begin{document}
\begin{center}
  
  \textbf{}

  \vspace{2.5cm}
  \Huge{}
  \textbf Tech Review
  \vspace{1.5cm}

 
  \LARGE
  CS461 - Senior Software Engr Project\\
  \vspace{0.25cm}
  Instructor: D. Kevin McGrath \\
  Instructor: Kristen Winters \\
  \vspace{0.25cm}
  Fall 2018 \\
  \vspace{1cm}
  
  \large{Matthew Jansen}\\
  \vfill
  November 19th, 2018\\
  \vspace{1cm}
  \vspace*{\fill}
   \begin{abstract}
       \noindent This document will discuss the different technologies used within our project, specifically how the project will export pictures and videos from within the web application. These technologies are a necessary component to the functionality and usability of our web application, as the project is supposed to allow the user to obtain snapshots or movies of their results and use them for presentation. To explain how the web application will give this feature to its users, we will first discuss the different formats of files that can be exported, such as PNG and GIF files for pictures and MOV and MP4 for movies. Finally, we will show how Javascript frameworks such as html2canvas.js, gif.js, whammy.js and ffmpegserver.js can be implemented to export the files to the application user.
   \end{abstract}
   \normalsize 
  \end{center}
  
\section{User Presentation of Export Module}
Before discussing how we will be exporting these pictures, a brief overview of the prerequisites for this functionality should be considered. Moving forward, when talking about the logistics of the webpage that contains the export module, we will make the following assumptions. \newline
\indent First, the user is able to view their 3D environment that contains all of their data points and general plot information. Second, the user is able to view an export module that allows them to select one or more frames/snapshots of their 3D environment that the user would like to be converted and exported as a picture/movie file. Third, an “export” button is available to the user, which the user can press after selecting one or more frames/snapshots to convert and export. After selecting this button, the user can expect that the web application will convert the selected frames/snapshots into the picture/movie type that they choose. \newline
\indent For pictures, the user can choose between a PNG file type for a single picture, or a GIF file type for more than one picture selected. As for movie files, after the user selects more than one frame/snapshot, the user can choose between downloading a movie file as a MP4 file type, or as a MOV file type. After these selections are made and the user clicks the “export” button, the user can expect the web application to download the respective file using their web browser. \newline

\section{Exporting Pictures}
With the ability to export pictures, the user is able to contain a summary of their desired data points in a 3D environment, such that a customer presentation can be done without the need to be online.

\subsection{Picture file types}
In this web application, the module to export images of the 3D environment will support PNG and GIF file types. PNG file types are universally used, along with GIF file types, for single-image formats (PNG), as well as animations that span over several images (GIF).	

\subsection{html2canvas.js}
html2canvas\cite{html2canvas} is a javascript module which allows a user to screenshot a specific frame of a web browser. This happens because the scripts render a screenshot by reading the DOM and the different elements styles applied to the elements. To do this, the html2canvas() function is called within the web application if the user wishes to download only a single-frame image (PNG file type). While this happens, two parameters will need to be passed in: the document’s body (document.body), and the properties of the screenshot. The properties of the screenshot will be a set of characteristics that may be set to customize the layout of the picture (for example, picture height or width). \newline
\indent In order to download this screenshot, we will need to create a link set with our image data type, and we will do this using setting the ‘href’ attribute of the link to use the canvas.toDataURL(“image/png”) method within our web application. After this, setting the ‘download’ attribute to a specific name will create a name for our picture as the user downloads it. Lastly, we need to trigger the link using the click() method (NOTE: a ‘[0]’ must be appended before applying the click() method, as we are using the click() native method. For example, “\$(‘test’)[0].click()” ). Together, these methods will provide the webpage the functionality to download PNG images of the user’s 3D environment in a single-frame format.

\subsection{gif.js}
The gif.js\cite{gif.js} framework is a javascript framework which encodes GIFS, and uses typed arrays and web workers to render frames. This functionality gives gif.js the ability to perform GIF encoding very quickly, and all in the background of the web browser. This framework will come into play when the user wishes to save one or more snapshots into the form of a GIF animation, rather than a movie file. As we saw in the html2canvas section, we can create a snapshot of the current 3D environment using the html2canvas(), canvas.toDataURL(), and click() methods within our web application. After saving these snapshot(s) that the user would like to convert into a GIF, we can pass them into the gif.js framework using the gif.addFrame() method, which can take in parameters of either image elements or canvas elements. Once we finish adding frames into the GIF, we can use the gif.render() method to render the gif, and then download it to the user’s web browser. Doing this gives the user the ability to save animations, alongside image and movie files, giving the user a wider array of options to choose from. 
\newline

\section{Exporting Movie Files}
Having the ability to export movie files allows the user to explore more options when deciding how the data should be presented to a group of people. Additionally, this allows the user to not just be limited to image formats if high quality graphics are desired when displaying more than one snapshot of the 3D environment. 

\subsection{Movie file types}
In this web application, the module to export videos of the 3D environment will support MOV and MP4 file types. These two movie file types are used universally for playing videos, so if a user is able to play videos on their device, then chances are that they will be able to export videos from this web application.

\subsection{whammy.js}
In order for the web application to export MOV and MP4 file types, we must first convert the snapshots that we obtain using the html2canvas() methods into a movie format. To do this, we will use the whammy.js\cite{whammy.js} framework to convert these canvas elements into a webm format. \newline
\indent This can be done by first defining a new whammy video, var encoder = new Whammy.Video(x), where x is the number of frames we wish to add into the video. Next, we can begin adding canvas elements into the whammy video, encoder.add(canvasElement), where canvasElement is the canvas element that we obtained through html2canvas. Once we are finished adding snapshots of the 3D environment into the video, we can compile the frames, encoder.compile(false, function(output){}). This will create a webm file type which contains the compiled snapshots that we wish to turn into a video. The next steps include converting this webm type file into a MOV or MP4 file type, which we will do using ffmpegserver.js. 

\subsection{ffmpegserver.js}
This javascript framework, ffmpegserver.js\cite{ffmpegserver.js} is a simple node server that begins a capture, collects canvas frames, and converts them to a MP4 format. To use it, a CCapture object needs to first be created using var capturer = new CCapture ( { format: ‘ffmpegserver’ } ); Depending on the user’s choice of video format, we can choose to do a more in-depth definition of the capturer object and choose to capture the frames as a MOV file type, however the default file type when defined as stated above is a MP4 file. After the object is defined, we can begin capturing using the capturer.start() method. Next, we define a render() function which contains the capturer.capture(canvasElement) method, which will collect the canvas elements (collected using html2canvas) into the capturer object. Lastly, we can call capturer.stop() to cease the collection of canvas elements, and then capturer.save() to save the captured frames as the movie file type which was specified earlier by the user. After this is done, the web application can begin downloading the movie file using the user’s browser.

\section{Conclusion}
In conclusion, through the use of four javascript frameworks: html2canvas, gif.js, whammy.js, and ffmpegserver.js, we are able to add the functionality of exporting snapshots of the 3D environment from the web application to the client. Furthermore, we are able to do so in a multitude of file formats, ranging from PNGs and GIFs to MOV and MP4 movie file types. This gives the customer a wide range of options to choose from, and can be very beneficial when deciding the best method to present the data. 

\newpage  

\bibliographystyle{IEEEtran}
\bibliography{ref.bib}
\end{document}